\documentclass[11pt]{article}
\usepackage{colacl}
\title{{\bf Memory-Based Named Entity Recognition}}
\author{
\begin{tabular}{cc}
{\bf Erik F. Tjong Kim Sang}\\
CNTS - Language Technology Group\\
University of Antwerp\\
{\it erikt@uia.ua.ac.be}
\end{tabular}
}

\date{\today}

\begin{document}
\maketitle

\section{Introduction}

We apply a memory-based learner to the CoNLL-2002 shared task:
language-independent named entity recognition.
We use three additional techniques for improving the base performance
of the learner: cascading, feature selection and system combination.
The overall system is trained with two types of features: words and
substrings of words which are relevant for this particular task.
It is tested on the two language pairs that were available for this
shared task: Spanish and Dutch.

\section{Approach}

In this section we will give a brief description of the basic
techniques employed in our approach to named entity recognition.
We will describe memory-based learning, cascading, feature selection
and system combination.

We use a nearest neighbor memory-based learner as a basic classifier
\cite{timbl2001}.
The learner stores all training data and classifies new data items by
comparing them with the training data.
The new data item will receive the same classification as that of the
training item that is most similar to it.
The data items are represented with symbolic features for which
the learner computes weights which are based on their relevance for
the classification task.
The memory-based software package we used is called Timbl
\cite{timbl2001}.
We use its default learning algorithm, instance-based learning with
information gain weighting ({\sc ib1ig}), with the default setting
of parameters (for example k=1).

The task of the learner is to predict the positions of the named
entities in a text.
The entities have been encoded with so-called IOB tags.
These are tags which show that a word is outside of any entity (O),
inside an entity (I) or at the beginning of an entity (B).
For example, the sentence {\it John Smith called .} has the associated
tags B-PER I-PER O O.
This means that {\it John} starts a named entity of type PER, {\it
Smith} continues this entity and neither {\it called} nor the final
period are part of a named entity.
The task of a system processing this data is to predict the sequences
of entity tags as well as possible.

The initial set of parameters which we use for the learner for
predicting the best entity tag for a word consists of the word and a
group of preceding and following words.
For example, we could use {\it John}, {\it Smith} and {\it called}
as features when trying to find the best tag for {\it Smith} in the
example sentence.
In that case we would be using a left context of one word (in this
case {\it John}) and a right context of one word (here {\it called}).
It would be useful to know if the context words could be part of a
named entity as well.
In order to obtain this information, we perform {\sc cascading},
feeding the output of one learner to the input of another learner
\cite{veenstra98}.
We will train a classifier on this task by using basic features such
as words and then use the output tags of this system as input features
for a second learner.
For practical reasons we will only use the class tags of the context
words and not that of the focus word.
For example, the second system could represent the word {\it Smith}
in the example sentence with five features: 
John, Smith, called, B-PER and O.

In general using features from a large context will give more
detailed information about a word.
However, the performance of memory-based learners can suffer when they
need to process data with many features \cite{tks2002jmlr}.
Since we do not know what word context features are best for this
task, we will attempt to find the best set automatically by performing
{\sc feature selection}.
There are too many different feature sets to perform a complete
search.
We will use a search method called bi-directional hill-climbing
\cite{caruana94} for exploring the feature space.
This method starts from a set of features (in our case the empty set)
and compares the performance of a learner using this set with learners
using the set with an extra feature or with one feature less.
When the algorithm finds a feature set that enables the learner to
perform better then it performs another search with this feature set.
This procedure is repeated until the performance of the current
feature set cannot be improved.

In our work on the CoNLL-2000 shared task of chunking
\cite{tks2002jmlr}, we have shown that performance on a phrase
classification task can be improved by performing {\sc system
combination}.
Since named entity classification is similar to chunking, we will use
this technique here as well.
We will use the same approach as described in Tjong Kim Sang
\shortcite{tks2002jmlr}: apply one learning technique to five
different representations of the output tags: IOB1, IOB2, IOE1, IOE2
and O+C.
Changing the output representation will change the task of the
learner.
For different output representations it will make different errors.
We will convert the output for the different data representations to 
one data representation (O+C) and for each word select the
tag that has been predicted most often (majority voting).

In \cite{tks2002jmlr}, we have evaluated three different processing
strategies for finding chunks in text: 1. predicting chunk boundaries
and chunk classes simultaneously, 2. predicting boundaries first and
classes later, and 3. building a separate recognizer for each
different class.
We chose the second processing strategy because it required fewer 
computer resources than the third and performed better than the first.
We use this approach here as well: first we attempt to find 
the boundaries of named entities and then we compute the most
likely class for the entities that have been found.

\begin{table*}[t]
\begin{center}
\begin{tabular}{|l|c|ll|c|lll|}\cline{2-8}
\multicolumn{1}{l|}{Spanish train} & \multicolumn{3}{c|}{Pass 1}  
                           & \multicolumn{4}{c|}{Pass 2} \\\hline
Representation& F$_{\beta=1}$ & \multicolumn{2}{c|}{features used} & 
       F$_{\beta=1}$ & \multicolumn{3}{c|}{features used} \\\hline
IOB1 & 85.86 & w$_{-2..0}$                    & m$_{fp,fs,ps}$ 
     & 88.68 & w$_{-2,0,1}$   & t$_{-1,1}$    & m$_{fp,fs,ps}$ \\
IOB2 & 82.14 & w$_{-2..1}$                    & m$_{fs,pp,ps}$
     & 84.39 & w$_{-1..1}$    & t$_{-2,-1,1}$ & m$_{pp,ps}$    \\
IOE1 & 85.86 & w$_{-2..0}$                    & m$_{fp,fs,ps}$
     & 88.76 & w$_{-2,0,1}$   & t$_{-1,1}$    & m$_{fp,fs,ps}$ \\
IOE2 & 77.18 & w$_{-2..2}$                    & m$_{fp,fs,pp}$
     & 83.50 & w$_{-1,0,2}$   & t$_{-1,1}$    & m$_{fs,pp}$    \\
O+C  & 80.33 & O: w$_{-2..0}$                 & m$_{fp,pp,ps}$             
     & 84.08 & O: w$_{-2..0}$ & t$_{-2,-1}$   & m$_{fp,pp,ps}$ \\
     &       & C: w$_{-2..1}$                 & m$_{ps}$
     &       & C: w$_{-1..2}$ & t$_{1,2}$     & m$_{ps}$       \\\hline
Majority voting & 86.10 & \multicolumn{2}{l|}{O: IOB1 IOB2 IOE1 IOE2 O}
                & 88.96 & \multicolumn{3}{l|}{O: IOB1 IOB2 IOE1 IOE2 O} \\
                &       & \multicolumn{2}{l|}{C: IOB1 IOB2 IOE1 IOE2 C}
                &       & \multicolumn{3}{l|}{C: IOB1 IOB2 IOE1 IOE2 C}\\\hline
with classes    & 72.29 & w$_{{\rm -2,-1,start,final}}$ & 
                          m$_{start.s}$
                & 74.34 & \multicolumn{2}{l}{w$_{{\rm -2,-1,start,final}}$}
                        & m$_{start.s}$\\\hline
\end{tabular}
\label{tab-wordsandmorph}
\caption{
F$_{\beta=1}$ rates for identifying entity borders (not entity types)
with word features (w) and morphological features (m, see text below)
only from the Spanish training data, processing with 10-fold
cross-validation while five different output data representations
(IOB1, IOB2, IOE1, IOE2 and O+C), cascading with extra classification
tag features (t), feature selection and system combination. 
The best results are obtained by using only a limited number of the
available features (w$_{-3..3}$, t$_{-2,-1,1,2}$ and
m$_{fp,fs,pp,ps}$). 
Cascading (pass 2) generally improves performance when compared
with pass 1.
Majority voting performs better than any of the individual learners
while using only a few of their results.
The bottom line shows the performance after adding class information.
}
\end{center}
\end{table*}

\section{Results}

In our first approach to named entity recognition, we have applied
the chunker described in Tjong Kim Sang \shortcite{tks2002jmlr} to the
Spanish data.
This data set did not have part-of-speech tags available so we have
only used words as features.
Each word was represented by itself and the three preceding and the
three next words.
We determined the best parameters for our approach by performing
experiments with the training data for Spanish while using 10-fold
cross-validation.
For this purpose the data was divided in ten parts of approximately
the same size and each part was processed while using the other nine
as training data.

We started with removing the entity class information from the data,
keeping entity borders only.
For each of the five available output representations we have
performed a feature selection process for finding an optimal set
of features for this task.
Each of the results was fed to cascaded system which had access to the
seven word features as well as the predicted class tags for the two
words before the focus word and the two words following the focus
word.
The results of the five cascaded systems were converted to brackets
(O+C representation) and these were combined with majority voting.
We evaluated all combinations of three, four and five systems and
choose the best.
Finally, classes were added to the resulting entities by a learner
which had access to the first and the last word of the entity plus
three words before the first word and three words behind the last.
Again, feature selection was used for finding the best feature set.
The output of the learners was evaluated with F$_{\beta=1}$ rates
which are based on the precision and recall of entities
\cite{vanrijsbergen75}.

In almost all cases the feature selection method used only a subset of
the available features (seven word features and four additional tag
features).
The cascaded systems outperformed the base systems in three of the
five cases.
A majority vote of the result was always better than the best of the
individual systems (at best F$_{\beta=1}$=79.40 for the cascaded
systems).
After determining the categories of the named entities, performance
dropped to F$_{\beta=1}$=71.45.

A problem of our current approach is that the system has few clues
for handling new words.
In the CoNLL-2000 chunking task, the part-of-speech tags helped to
classify these but in the Spanish data there is no additional
information about the words available except from their context.
It seems reasonable to use word-internal morphological information as
a clue.
This could be done by using the first few characters or the last few
characters of a word as an extra feature.
The problem is that we do not know how many characters we have to
select to get useful features.
Some character sequences of a specific length might be interesting for
this task while others of the same length might be not.
We have decided to use a statistical measure for selecting
morphological strings that are useful for performing this task in a
particular language.

The statistical measure which we have chosen selects all prefix and
suffix strings that appear in the training data in capitalized words
ten times or more and additionally are part of a word in a named
entity in 95\% of the cases or more. 
Examples of such strings in the Spanish data are {\it Europea}
(prefix, appears 61 times and is a part of named entity words in 98\%
of the cases) and {\it z} (suffix, appears 734 times and 99\% of the
time in named entity words).
The system looked for phrases in the Spanish training data of ten
characters or shorter and found 790.
The word immediately in front of a named entity word can be an important
clue and therefore we have extracted similar prefix and suffix
features for words immediately before capitalized named entity words.
Examples from the Spanish training data are the prefix {\it una}
(24 times, 100\%) and the suffix {\it e} (5042 times, 100\%).
For this particular type, 214 strings were selected.

We have added four morphological features to the data: focus word
prefix (fp), focus word suffix (fs), previous word prefix (pp) and
previous word suffix (ps).
After this we repeated the 10-fold cross-validation experiment with
the Spanish training data.
For each of the 10 parts, the morphological features were generated
from the other nine parts only.
The results of this experiment can be found in Table 1.
Morphological features were chosen as useful features in all cases.
The most frequently chosen feature was the suffix of the previous
word (ps).
The maximum performance of the unlabeled task when compared with using
word features only, improved considerably, from 79.40 to 88.96 (46\%
error decrease). 
The increase after adding categories to the named entities was
smaller: from 71.45 to 74.34 (10\% error decrease).

We have used the best configuration found for the Spanish training
data for processing both the Spanish test sets and a configuration
obtained from the Dutch training data (without the part-of-speech
tags) for the Dutch test sets.
The results for processing the test data sets can be found in Table
\ref{tab-final}.
Overall precision is always higher than overall recall but the
difference is never larger than 4.4 percentage points.
For both languages the system performs better on the test data than
on the development data.

\section{Concluding Remarks}

We have presented a machine learning method for performing 
language-independent named entity recognition.
It uses a memory-based classifier as base learner.
The performance of this learner is improved with cascading, feature
selection and system combination.
The system uses both words and prefixes and suffixes of words.
The latter two are derived with a statistical method which selects
substrings of words which frequently appear in words in or near
named entities.
The learner had no access to linguistic information other than that
was made available in the training data.
Its performance was not as good as state-of-the-art named entity
recognizers for English (over F$_{\beta=1}$=90, see for example
Mikheev \shortcite{mikheev98}).
However, it performs reasonable on the two languages in this shared
task (F$_{\beta=1}$=75 for the Spanish test set and F$_{\beta=1}$=70
for Dutch).

The strength of our system is its ability to operate without many
linguistic clues about the language that is processed.
A text with annotated entities is enough to obtain a reasonable
performance.
A practical weakness is its processing speed: the current
implementation processes only about 6 words per second on a parallel
machine. 
Another weakness is the selection of morphological features.
This relies on the fact that many named entity words in the two target
languages are capitalized, a feature which may not help for other
languages (for example German and Hindi).
We believe that a further improvement of the performance of the system
could be obtained by using features derived from interesting statistical
information from the training text and perhaps even from other 
untagged text.


\section*{Acknowledgements}

Tjong Kim Sang is supported by IWT STWW as a researcher in the ATRANOS 
project.

\begin{table}[t]
\begin{center}
\begin{tabular}{|l|c|c|c|}\cline{2-4}
\multicolumn{1}{l|}{Spanish dev.}
                 & precision & recall & F$_{\beta=1}$ \\\hline
LOC     & 67.90\% & 81.83\% & 74.22 \\
MISC    & 51.76\% & 42.92\% & 46.93 \\
ORG     & 77.08\% & 70.24\% & 73.50 \\
PER     & 86.90\% & 77.09\% & 81.70 \\\hline
overall & 74.79\% & 71.99\% & 73.36 \\\hline
\multicolumn{4}{c}{}\\\cline{2-4}
\multicolumn{1}{l|}{Spanish test}
                 & precision & recall & F$_{\beta=1}$ \\\hline
LOC     & 76.01\% & 76.01\% & 76.01 \\
MISC    & 63.70\% & 50.59\% & 56.39 \\
ORG     & 76.45\% & 78.36\% & 77.39 \\
PER     & 79.57\% & 81.09\% & 80.32 \\\hline
overall & 76.00\% & 75.55\% & 75.78 \\\hline
\multicolumn{4}{c}{}\\\cline{2-4}
\multicolumn{1}{l|}{Dutch devel.}
                 & precision & recall & F$_{\beta=1}$ \\\hline
LOC     & 79.21\% & 72.06\% & 75.47 \\
MISC    & 68.63\% & 65.68\% & 67.12 \\
ORG     & 76.13\% & 49.78\% & 60.20 \\
PER     & 62.61\% & 80.65\% & 70.49 \\\hline
overall & 69.60\% & 66.77\% & 68.15 \\\hline
\multicolumn{4}{c}{}\\\cline{2-4}
\multicolumn{1}{l|}{Dutch test}
                 & precision & recall & F$_{\beta=1}$ \\\hline
LOC     & 83.21\% & 73.28\% & 77.93 \\
MISC    & 72.79\% & 64.45\% & 68.36 \\
ORG     & 75.63\% & 54.50\% & 63.35 \\
PER     & 65.72\% & 81.97\% & 72.95 \\\hline
overall & 72.56\% & 68.88\% & 70.67 \\\hline
\end{tabular}
\end{center}
\caption{
Results obtained for the development and the test data sets for the
two languages used in this shared task.
} 
\label{tab-final}
\end{table}

\small

\bibliographystyle{acl}
\bibliography{ref}

\end{document}

\begin{table*}[t]
\begin{center}
\begin{tabular}{|l|c|l|c|ll|}\cline{2-6}
\multicolumn{1}{l|}{Spanish train} & \multicolumn{2}{c|}{Pass 1}  
                           & \multicolumn{3}{c|}{Pass 2} \\\hline
Representation& F$_{\beta=1}$ & \multicolumn{1}{c|}{features used} & 
       F$_{\beta=1}$ & \multicolumn{2}{c|}{features used} \\\hline
IOB1 & 77.12 & w$_{-2..0}$ 
     & 77.12 & w$_{-2..0}$ & \\
IOB2 & 73.48 & w$_{-2..0}$ 
     & 74.90 & w$_{-2..0}$ & tag$_{-1}$\\
IOE1 & 77.33 & w$_{-2..0}$ 
     & 77.33 & w$_{-2..0}$ & \\
IOE2 & 73.71 & w$_{-2..0}$ 
     & 75.26 & w$_{-2..0}$ & tag$_{1}$\\
O+C  & 73.21 & O: w$_{-3..1}$              
     & 75.12 & O: w$_{-2..1}$ & tag$_{-2,-1,1}$ \\
     &       & C: w$_{-2..1}$ 
     &       & C: w$_{-2..1}$ & tag$_{-2,-1,1}$ \\\hline
Majority voting & 75.70 & O: IOB1 IOB2 IOE1 IOE2 O             
                & 79.40 & \multicolumn{2}{l|}{O: IOE1 IOE2 O} \\
                &       & C: C 
                &       & \multicolumn{2}{l|}{C: IOB1 IOB2 IOE1 C} \\\hline
with categories & 68.15 & w$_{{\rm -1,start,final}}$
                & 71.45 & \multicolumn{2}{l|}{word$_{{\rm -1,start,final}}$}\\\hline
\end{tabular}
\label{tab-wordsonly}
\caption{
F$_{\beta=1}$ rates for identifying entity borders (not entity types)
with word features (w) only from the Spanish training data, processing
with 10-fold cross-validation while five different output data
representations (IOB1, IOB2, IOE1, IOE2 and O+C), cascading with extra
classification tag features (t), feature selection and system
combination.
The best results are obtained by using only a limited number of the
available features (word$_{-3..3}$ and tag$_{-2,-1,1,2}$).
Cascading (pass 2) generally improves performance when compared
with pass 1.
Majority voting performs better than any of the individual learners
while using only a few of their results.
Note that the bracket representation O+C (open and close) consists
of two data streams which have been processed separately.
The bottom line shows the performance with an extra classification 
pass which identified the types of the entities.
}
\end{center}
\end{table*}

