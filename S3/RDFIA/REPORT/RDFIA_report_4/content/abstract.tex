\vspace{1cm}
\begin{abstract}

The goal of this practical work is to understand Bayesian Deep Learning, which is split into three parts. We begin with a quick introduction to linear regression, then we examine approximate inference, and finally, we examine the applications of uncertainty. Usually, deep learning models output a single prediction, but in contrast, the bayesian counterpart outputs a probability distribution over possible outputs. This allows it to represent uncertainty in its predictions, which is particularly useful for modeling uncertainty that is often present in real-world data.

\end{abstract}
